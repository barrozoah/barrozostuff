% This text is proprietary.
% It's a part of presentation made by myself.
% It may not used commercial.
% The noncommercial use such as private and study is free
% Sep. 2005 
% Author: Sascha Frank 
% University Freiburg 
% www.informatik.uni-freiburg.de/~frank/


\documentclass{beamer}
\usetheme{bjeldbak}
\usepackage{xcolor}
\usepackage{graphicx}
\usepackage[export]{adjustbox}

\usepackage{tcolorbox}
\usepackage{tabularx}
\usepackage{array}
\usepackage{colortbl}
\tcbuselibrary{skins}

\newcolumntype{Y}{>{\raggedleft\arraybackslash}X}

\tcbset{tab1/.style={fonttitle=\bfseries\large,fontupper=\normalsize\sffamily,
colback=pale,colframe=kindaorange,colbacktitle=white,
coltitle=black,center title,freelance,frame code={
\foreach \n in {north east,north west,south east,south west}
{\path [fill=kindaorange] (interior.\n) circle (3mm); };},}}

%\tcbset{tab2/.style={enhanced,fonttitle=\bfseries,fontupper=\normalsize\sffamily,
%colback=yellow!10!white,colframe=red!50!black,colbacktitle=Salmon!40!white,
%coltitle=black,center title}}


\definecolor{goldenrod}{HTML}{CDC279}
\definecolor{yellowgreen}{HTML}{A6CD7F}
\definecolor{pale}{HTML}{F0E9D9}
\definecolor{kindaorange}{HTML}{FFBD52}

\begin{document}
\title{Electrostatic Flexibility in the Alkaline Phosphatase Superfamily}   
\author{Alexandre H. Barrozo} 
\date{\today} 

\frame{\titlepage} 

\frame{\frametitle{Overview}\tableofcontents} 


\section{Catalytic promiscuity in the AP superfamily} 
%\subsection{Alkaline}
\frame{\frametitle{The alkaline phosphatase and its substrates} 
\begin{figure}
\vspace*{-0.45cm}
\centering
\includegraphics[width=0.825\textwidth,keepaspectratio]{images/ap-substrates.png}
\end{figure}
}

\frame{\frametitle{Crosswise catalytic promiscuity}
\begin{figure}
\vspace*{-0.5cm}
\hspace*{0.25cm}
\includegraphics[width=0.8\textwidth,keepaspectratio,left]{images/crossreactivity.png}
\end{figure}
}

\frame{\frametitle{Comparing active sites}
\begin{figure}
\vspace*{-0.125cm}
\centering
\includegraphics[width=0.825\textwidth,keepaspectratio]{images/activesites.png}
\end{figure}
}

\frame{\frametitle{\textcolor{goldenrod}{\textbf{\textit{Rl}PMH}} X \textcolor{yellowgreen}{\textbf{PAS}}}
\begin{figure}
\vspace*{-0.15cm}
\centering
\includegraphics[width=0.75\textwidth,keepaspectratio]{images/as-overlay.png}
\end{figure}
}

\section{Phosphonate Monoester Hydrolase} 
\frame{\frametitle{Phosphonate monoester hydrolase}
\begin{figure}
\vspace*{-0.75cm}
\hspace*{-0.75cm}
\centering
\includegraphics[width=1.125\textwidth,keepaspectratio]{images/pmh-substrates.png}
\end{figure}
}

\frame{\frametitle{The active site}
\begin{figure}
\vspace*{-0.125cm}
\hspace*{-0.5cm}
\centering
\includegraphics[width=0.75\textwidth,keepaspectratio]{images/pmh-as.png}
\end{figure}
}

\frame{\frametitle{Suggested mechanism}
\begin{figure}
\vspace*{-0.25cm}
\hspace*{-0.75cm}
\centering
\includegraphics[width=1.15\textwidth,keepaspectratio]{images/pmh-mech2.png}
\end{figure}
}

\frame{\frametitle{Energetics for the hydrolysis of \textcolor{blue}{\textit{Rl}PMH} and \textcolor{red}{\textit{Bc}PMH}}
\begin{figure}
%\vspace*{-0.25cm}
\centering
\includegraphics[width=0.9\textwidth,keepaspectratio]{images/energiessubstrates.png}
\end{figure}
}

\frame{\frametitle{Energetics for the hydrolysis of \textcolor{blue}{PHOSPHONATE} and \textcolor{red}{PHOSPHODIESTER}}
\begin{figure}
\hspace*{-0.25cm}
\centering
\includegraphics[width=0.875\textwidth,keepaspectratio]{images/energiesmutants.png}
\end{figure}
}

\frame{\frametitle{Electrostatic contributions for every mutant}
\begin{figure}
\vspace*{-0.125cm}
\hspace*{-0.25cm}
\centering
\includegraphics[width=1.0\textwidth,keepaspectratio]{images/gc-2vqr-mutants.png}
\end{figure}
}

\section{Analysis of Distinct Active Sites} 
\frame{\frametitle{Different shapes}
\begin{figure}
\vspace*{-0.15cm}
\centering
\includegraphics[width=0.75\textwidth,keepaspectratio]{images/ap-pocketsshape.png}
\end{figure}
}

\frame{\frametitle{Polarity of the pockets}
\begin{figure}
\vspace*{-0.15cm}
\centering
\includegraphics[width=0.75\textwidth,keepaspectratio]{images/ap-pocketspolarity.png}
\end{figure}
}

\frame{\frametitle{Active sites properties}
\begin{center}
\tcbset{center title}
\tcbox[left=0mm,right=0mm,top=0mm,bottom=0mm,boxsep=0mm,
toptitle=0.5mm,bottomtitle=0.5mm,title=Active site properties]{%
\arrayrulecolor{kindaorange}\renewcommand{\arraystretch}{1.2}%
\begin{tabular}{ c c c c }
\small
\centering

\textbf{Enzyme} & \textbf{Volume} & \textbf{Polar SA} & \textbf{Activities} \\
AP & 1870.7 & 320 & 6 \\
PMH & 1215 & 240.4 & 5 \\
NPP & 1016.9 & 238.5 & 5 \\
PAS & 756.7 & 87.2 & 3 \\
ASA & 540.7 & 181.3 & 2 \\
N-ac4S & 431.9 & 106.6 & 2 \\
iPGM & 151.2 & 39.2 & 1 \\

\end{tabular}}
\end{center}
}

\section{Conclusions}
\frame{\frametitle{Conclusions}
\begin{itemize}
   \item Electrostatic flexibility
   \begin{itemize}
      \item Five reactions, one single mechanism
      \item Sturdy active site
   \end{itemize}
   \item Possible trend between activities, volume and polar surface area
\end{itemize}
}

\frame{\frametitle{Great people to thank and buy a dinner}
\begin{itemize}
   \item Lynn Kamerlin
   \item Fernanda Duarte
   \item Alexandra Carvalho
   \item Paul Bauer
   \item Johan \r{A}qvist
\end{itemize}
}

\end{document}

